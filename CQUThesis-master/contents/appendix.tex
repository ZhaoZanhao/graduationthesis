
\chapter{附\hskip\ccwd{}录}


\section{程序源代码}

R实现多期二叉树:
\begin{lstlisting}
# 实现二叉树定价

# 安装fOptions包
#install.packages("fOptions")
#install.packages("timeDate")
#install.packages("timeSeries")
#install.packages("fBasics")

# 加载fOptions包
library(timeDate)
library(timeSeries)
library(fBasics)
library(fOptions)

CRRBinomialTreeOption(TypeFlag='ce', S=100, X=110,
 Time=6/12, r=0.05, b=0.05, sigma=0.2, n=3,
 title="二叉树定价")

# BS模型定价
GBSOption(TypeFlag = 'c', S=100, X=110, Time=6/12, r=0.05, 
b=0.05, sigma = 0.2, title="Black-Schools定价" )@price


# 实现3期二叉树定价
CRRTree=BinomialTreeOption(TypeFlag = 'pa', S=100, X=110, 
Time=6/12, r=0.05, b=0.05, sigma=0.2, n=3)

BinomialTreePlot(CRRTree, dy=1, cex=0.8, ylim=c(-5,5), 
xlab='n期', ylab='期权价值', main='选择树')


# 实现20期二叉树定价
CRRTree=BinomialTreeOption(TypeFlag = 'pa', S=100, X=110, 
Time=6/12, r=0.05, b=0.05, sigma=0.2, n=20)

BinomialTreePlot(CRRTree, dy=1, cex=0.8, ylim=c(-22,22), 
xlab='n期', ylab='期权价值', main='选择树')
\end{lstlisting}


SAS实现多期二叉树:
\begin{lstlisting}	
proc iml;
/* 看涨期权的定价公式 */
start call(i,sigma,t,S,X,n);
R=exp(i*t/n);
u=exp(sigma*sqrt(t/n));
d=exp(-sigma*sqrt(t/n));
q=(r-d)/(u-d);
index=(n:0);
matrix=j(n+1,3,.);
do i=1 to n+1;
matrix[i,1]=comb(n,index[i]);
matrix[i,2]=q**index[i]*(1-q)**(n-index[i]);
matrix[i,3]=max(S*u**index[i]*d**(n-index[i])-X,0);
end;
c=matrix[,#][+,] / R ** n;
return(c);
finish call;

/* 看跌期权的定价公式 */
start put(i,sigma,t,S,X,n);
R=exp(i*t/n);
u=exp( sigma * sqrt(t/n) );
d=exp(-sigma * sqrt(t/n) );
q=(r-d) / (u-d);
index=(n:0);
matrix=j(n+1,3,.);
do i=1 to n+1;
matrix[i,1]=comb(n,index[i]);
matrix[i,2]=q**index[i]*(1-q)**(n-index[i]);
matrix[i,3]=max(X-S*u**index[i]*d**(n-index[i]),0);
end;
p=matrix[,#][+,] / R ** n;
return(p);
finish put;

/* 调整n的大小,并计算看涨期权和看跌期权 */
do n=5 to 1000 by 5;
call=j(nrow(n),1,.);
put=j(nrow(n),1,.);
mattrib call format=5.3 put format=5.3;
do k=1 to nrow(n);
call[k]=round(call(0.05,0.1,1,100,100,n[k]),0.001);
put[k]=round(put(0.05,0.1,1,100,100,n[k]),0.001);
end;
print n call put;
end;

/* 调整i的大小,并计算看涨期权和看跌期权 */
do i=0.002 to 0.1 by 0.002;
call=j(nrow(i),1,.);
put=j(nrow(i),1,.);
mattrib i format=percent7.1 call format=5.3
 put format=5.3;
do k=1 to nrow(i);
call[k]=round(call(i[k],0.1,1,100,100,1000),0.001);
put[k]=round(put(i[k],0.1,1,100,100,1000),0.001);
end;
print i call put;
end;

/* 调整sigma的大小,并计算看涨期权和看跌期权 */
do sigma=0.01 to 0.5 by 0.01;
call=j(nrow(sigma),1,.);
put=j(nrow(sigma),1,.);
mattrib sigma format=percent7.1 call format=5.3 
put format=5.3;
do k=1 to nrow(sigma);
call[k]=round(call(0.05,sigma[k],1,100,100,1000),0.001);
put[k]=round(put(0.05,sigma[k],1,100,100,1000),0.001);
end;
print sigma call put;
end;

/* 调整t的大小,并计算看涨期权和看跌期权 */
do t=0.1 to 5 by 0.1;
call=j(nrow(t),1,.);
put=j(nrow(t),1,.);
mattrib t format=5.1 call format=5.3 put format=5.3;
do k=1 to nrow(t);
call[k]=round(call(0.05,0.1,t[k],100,100,1000),0.001);
put[k]=round(put(0.05,0.1,t[k],100,100,1000),0.001);
end;
print t call put;
end;

/* 调整x的大小,并计算看涨期权和看跌期权 */
do x=80 to 130 by 1;
call=j(nrow(x),1,.);
put=j(nrow(x),1,.);
mattrib x format=5.1 call format=5.3 put format=5.3;
do k=1 to nrow(x);
call[k]=round(call(0.05,0.1,1,100,x[k],1000),0.001);
put[k]=round(put(0.05,0.1,1,100,x[k],1000),0.001);
end;
print x call put;
end;
quit;
\end{lstlisting}

Python编写StockOption类:
\begin{Python}
	
	# -*- coding: utf-8 -*-
	"""
	Created on Fri May  3 10:08:18 2019
	
	@author: 赵赞豪
	"""
	
	# 编写StockOption类
	import math
	
	class StockOption(object):
	def __init__(self, S0, K, r, T, N, params):
	self.S0 = S0
	self.K  = K
	self.r  = r
	self.T  = T
	self.N  = max(1, N)
	self.STs= None
	
	self.pu          = params.get("pu", 0)
	self.pd          = params.get("pd", 0)
	self.div         = params.get("div", 0)
	self.sigma       = params.get("sigma", 0)
	self.is_call     = params.get("is_call", True)
	self.is_european = params.get("is_eu", True)
	
	self.dt = T/float(N)
	self.df = math.exp(-(r-self.div)*self.dt)
		
\end{Python}


Python编写BinomialEuropeanOption类:
\begin{Python}
	# -*- coding: utf-8 -*-
	"""
	Created on Fri May  3 10:46:08 2019
	
	@author: 赵赞豪
	"""
	
	
	from StockOption import StockOption
	import math
	import numpy as np
	
	class BinomialEuropeanOption(StockOption):
	def __setup_parameters__(self):
	self.M  = self.N + 1
	self.u  = 1 + self.pu
	self.d  = 1 - self.pd
	self.qu = (math.exp((self.r-self.div)*self.dt) - self.d) / (self.u-self.d)
	self.qd = 1-self.qu
	
	def _initialize_stock_price_tree_(self):
	self.STs = np.zeros(self.M)
	for i in range(self.M):
	self.STs[i] = self.S0*(self.u**(self.N-i))*(self.d**i)
	
	def _initialize_payoffs_tree_(self):
	payoffs = np.maximum(0, (self.STs - self.K) if self.is_call else (self.K - self.STs))
	return payoffs
	
	def _traverse_tree_(self, payoffs):
	for i in range(self.N):
	payoffs = (payoffs[:-1] * self.qu + payoffs[1:] * self.qd) * self.df
	return payoffs
	
	def __begin_tree_traversal__(self):
	payoffs = self._initialize_payoffs_tree_()
	return self._traverse_tree_(payoffs)
	
	def price(self):
	self.__setup_parameters__()
	self._initialize_stock_price_tree_()
	payoffs = self.__begin_tree_traversal__()
	return payoffs[0]
\end{Python}


















\section{附录的图和表}
以下内容用来测试附录中的插图和插表是否正常,主要的关注点在题注:

\begin{figure}[tbh]
\centering
\includegraphics[width=0.5\linewidth]{CQUbadge}
\caption{附录插图测试}
\label{fig:cqubadge}
\end{figure}

\begin{figure}[tbh]
	\centering
	\includegraphics[width=0.5\linewidth]{CQUbadge}
	\caption{附录插图测试}
	\label{fig:cqubadge2}
\end{figure}

\begin{table}[htb]
	\centering\colsep[24pt]
	\caption{本课题研究的两个自变量}
	\label{tab:inroVarible}
	\begin{tabularx}{\linewidth}{cl}
		\toprule
		\headcell{自变量} & \headcell{自变量可取的值} \\
		\midrule\setxuhao[6]
		是否接触大气 & \xuhao[1] 接触大气 \xuhao 氮气保护 \\\setxuhao[2]
		溶解方式 & \xuhao[1] 超声30min \xuhao 搅拌1h \xuhao 静置12h\\
		表格的第三行 & \bigcell{使用\cs{bigcell}\\可主动换行}\\
		\bottomrule
	\end{tabularx}
\end{table}

\begin{equation}
\alpha\beta\gamma\delta\epsilon\varepsilon\zeta\eta = AB\Gamma\varGamma Z
\end{equation}\eqlist{附录中的公式编号1,双语}[Equation name in English A]

\begin{equation}
\alpha\beta\gamma\delta\epsilon\varepsilon\zeta\eta = CD\Gamma\varGamma Z
\end{equation}\eqlist{附录中的公式编号2,双语}[Equation name in English B]

测试用途:theequation值为:\theequation ,thefigure值为:\thefigure ,thetable值为:\thetable
