

\begin{thebibliography}{1}
	\addcontentsline{toc}{section}{参考文献}
	\bibitem{cox}  Cox, J. C., J. E. Ingersoll, and S. A. Ross. An Intertemporal General Equilibrium Model of Asset Prices, Econometrica, 53(1985), 363-84.
	\bibitem{cox} Cox, J. C., S. A. Ross, and M. Rubinstein. Option Pricing: A Simplified Approach, Journal of Financial Economics, 7(October 1979), 229-64.
	\bibitem{cox} Cox, J. C., and M. Rubinstein. Option Markets. Englewood Cliffs, N. J.: Prentice Hall,1985.
	\bibitem{cox} Cox, J. C., S. A. Ross. The Valuation of Options for Alternative Stochastic Processes, Journal of Financial Economics, 3(1976), 145-66.
	\bibitem{black} Black, F. Fact and Fantasy in the Use of Options and Corporate Liabilities, Financial Analysts Journal, 31(July-August 1975), 36-41, 61-42.
	\bibitem{black} Black, F., and M. Scholes. The Valuation of Option Contracts and a Test of Market Efficiency, Journal of Finance, 27(May 1972), 399-418.
	\bibitem{black} Black, F., and M. Scholes. The Pricing of Options and Corporate Liabilities. Journal of Political Economy. 81(May-June 1973), 637-59.
	\bibitem{liu} 刘海洋. \LaTeX 入门 [M]. 北京: 电子工业出版社, 2013.
	\bibitem{hu}  胡伟. \LaTeX 2e完全学习手册(第二版)[M] 北京: 清华大学出版社, 2013.
	\bibitem{zhu}  朱世武.金融计算与建模理论、算法与SAS程序[M].北京:清华大学出版社,2007.
	\bibitem{xu}  许启发,蒋翠侠.R软件及其在金融定价分析中的应用[M].北京:清华大学出版社,2015.
	\bibitem{song}  宋军,张宗新.金融计量学-基于SAS的金融实证研究[M].北京:北京大学出版社,2009.
	\bibitem{cai}   (美)蔡瑞胸;王远林,王辉,潘家栋译.金融时间序列分析(第三版)[M].北京:人民大学出版社,2012.
	\bibitem{wu}  吴喜之.复杂数据的统计方法-基于R的应用(第二版)[M].北京:中国人民大学出版社,2013.
	\bibitem{ma}  (新) 马伟民;张永冀,霍达,张彤译.Python金融数据分析[M].北京:机械工业出版社,2018.
\end{thebibliography}